%----------------------------------------------------------------------------------------
%	PACKAGES AND OTHER DOCUMENT CONFIGURATIONS
%----------------------------------------------------------------------------------------

\documentclass[11pt]{developercv} % Default font size, values from 8-12pt are recommended

%----------------------------------------------------------------------------------------

\begin{document}

%----------------------------------------------------------------------------------------
%	TITLE AND CONTACT INFORMATION
%----------------------------------------------------------------------------------------

\begin{minipage}[t]{0.45\textwidth} % 45% of the page width for name
	\vspace{-\baselineskip} % Required for vertically aligning minipages
	
	\colorbox{black}{{\HUGE\textcolor{white}{\textbf{\MakeUppercase{André}}}}} % First name
	\colorbox{black}{{\HUGE\textcolor{white}{\textbf{\MakeUppercase{Fernandes}}}}} % Last name
	\vspace{6pt}
 
	{\fontsize{15}{15} \selectfont Desenvolvedor de Software} % Career or current job title
 
\end{minipage}
\hfill
\begin{minipage}[t]{0.275\textwidth} % 27.5% of the page width for the first row of icons 
	\icon{MapMarker}{12}{Florianópolis - SC}
	
	\icon{Phone}{12}{+55 (48) 991 673 945}
 
	\icon{At}{12}{\href{mailto:fpf.andre@gmail.com}{fpf.andre@gmail.com}}
 
	\icon{Github}{12}{\href{https://github.com/andrefpf}{github.com/andrefpf}}
\end{minipage}

\vspace{0.5cm}


%----------------------------------------------------------------------------------------
%	EXPERIENCE
%----------------------------------------------------------------------------------------

\cvsect{Experiência}

\begin{entrylist}
	\entry
		{2023 -- Hoje}
		{Desenvolvedor de Software}
		{Laboratório de Vibrações e Acústica (LVA) - UFSC}
		{
			Desenvolvimento de softwares de engenharia para simulação de pulsação acusticamente induzida em tubulações para Petrobrás utilizando Python. 
			
			\texttt{Python} / \texttt{Git} / \texttt{PyQt} / \texttt{VTK} / \texttt{Matplotlib}
		}

	\entry
		{2023 -- Hoje \\ \footnotesize{Voluntário}}
		{Iniciação Científica}
		{Embedded Computing Lab (ECL) - UFSC}
		{
			Desenvolvimento de algoritmos para solução do problema da taxa alvo em light fields estáticos codificados através do modo de transformada 4D de acordo com o padrão JPEG Pleno.

			\texttt{Python} / \texttt{Git} / \texttt{PyQt} / \texttt{VTK} / \texttt{Matplotlib}
		}

	\entry
		{2022 -- 2023 \\ \footnotesize{meio período}}
		{Estagiário em Desenvolvimento Backend}
		{Cognyte}
		{
			Desenvolvimento de API Rest para integração de microsserviços, atualização de sistemas legados e solução de problemas.
            
            \texttt{C++} / \texttt{Python} / \texttt{Git} / \texttt{Docker} / \texttt{Boost}
        }
	\entry
		{2019 -- 2022 \\ \footnotesize{meio período}}
		{Desenvolvedor de Software}
		{Laboratório de Vibrações e Acústica (LVA) - UFSC}
		{
			Desenvolvimento de softwares de engenharia para simulação de pulsação acusticamente induzida em tubulações para Petrobrás utilizando Python. 
			
            \texttt{Python} / \texttt{Git} / \texttt{PyQt} / \texttt{VTK} / \texttt{Matplotlib}
        }
	\entry
		{2017 -- 2018 \\ \footnotesize{meio período}}
		{Assistente Administrativo}
		{NSC TV}
		{
			Iniciei minhas atividades no período de transição da marca e pude participar desse processo.
			Lidei com o setor de assessoria e cobrança fornecendo informações e documentos e fui responsável pelo cadastro interno dos funcionários.
        }
\end{entrylist}

%----------------------------------------------------------------------------------------
%	EDUCATION
%----------------------------------------------------------------------------------------

\cvsect{Formação}

\begin{entrylist}
	\entry
		{2019 -- 2024}
		{Bacharel em Ciência da Computação}
		{\\Universidade Federal de Santa Catarina - UFSC}
		{
			Método para Controle de Taxa Alvo na Compressão de Light Fields Seguindo o Padrão JPEG Pleno Parte 2 – 4DTM.
		}
\end{entrylist}

%----------------------------------------------------------------------------------------
%	ADDITIONAL INFORMATION
%----------------------------------------------------------------------------------------

\begin{minipage}[t]{0.3\textwidth}
	\vspace{-\baselineskip} % Required for vertically aligning minipages
	\cvsect{Linguas}
 
	\textbf{Portuguese} - Nativo \\
	\textbf{English} - Fluente
\end{minipage}
\hfill
\begin{minipage}[t]{0.3\textwidth}
	\vspace{-\baselineskip} % Required for vertically aligning minipages
	\cvsect{Tecnologias}
 
    Python\\
    C++\\
    Javascript\\
	Rust\\
    Git\\
    Docker\\
    
    
\end{minipage}
\hfill
\begin{minipage}[t]{0.3\textwidth}
	\vspace{-\baselineskip} % Required for vertically aligning minipages
	\cvsect{Tópicos de Interesse}
 
    Codificação de Dados\\
    Computação Gráfica\\
    Simulações Físicas\\
	Linguagens Formais\\
    Open Source
\end{minipage}

%----------------------------------------------------------------------------------------

\end{document}
