\documentclass[11pt, a4paper]{moderncv}

\usepackage[brazil]{babel}
\usepackage{geometry}
\usepackage{mathpazo}
\renewcommand{\sfdefault}{ppl}

\moderncvstyle{casual}
\moderncvcolor{orange}
\usepackage[utf8]{inputenc}

\firstname{André}
\familyname{Fernandes}
\title{desenvolvedor de software}

\address{Florianópolis}{SC}
\mobile{(48) 991-673-945}
\email{fpf.andre@gmail.com}


\begin{document}
\makecvtitle{ }
\pagestyle{empty}

\section{Educação}
    \cventry
        {2019 -}
        {Bacharelado em Ciências da Computação}
        {}{}{}
        {Universidade Federal de Santa Catarina}

\section{Experiências profissionais}
    \cventry {2019 - 2022}
        {Bolsista}{Laboratório de Vibrações e Acústica (LVA) - UFSC}
        {}{}
        {
            Atuei como desenvolvedor python criando um software de análise de pulsação acústica em tubulações de gás natural em parceria com a Petrobrás.
        }
        
    \cventry {2017 - 2018}
        {Assistente Administrativo}{NSC TV}
        {}{}{}


\section{Programação}
    \cvitem{Linguagens}{Python, C, C++, Javascript, Arduino}
    \cvitem{Tecnologias}{GIT, Qt, VTK, Matplotlib, Linux, PostgreSQL, HTML, CSS}
    \cvitem{Interesses}{Compressão de Dados, Computação Gráfica, Desenvolvimento e Análise de Algoritmos}

\section{Línguas}
    \cvitem{Português}{Nativo}
    \cvitem{Inglês}{Avançado}
    
\section{Links}
    \cvitem{}{\url{https://github.com/andrefpf}}
    \cvitem{}{\url{https://www.linkedin.com/in/andrefpf/}}

    
    
\thispagestyle{plain}

\end{document}
